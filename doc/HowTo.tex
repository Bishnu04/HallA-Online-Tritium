\documentclass{report}
\begin{document}

\title{How To Set Up and Use The Tritium Replay Code}
\author{Tyler Hague}
\maketitle

\chapter{Introduction}
The official replay code for the tritium family of experiments lives in GitHub. This decision was made to allow better version control in a large collaboration. The main repository will have everything set up for working on the \textit{aonl} machines. This document will walk you through setting up your fork of this repository to work in your personal workspace.

This document assumes that you already have ROOT 6 and the Hall A Analyzer version 1.6 installed in your personal workspace.

\chapter{Setting Up The Code}

\section{Set Up Environmental Variables}

There are four environmental variables that need to be set in order for the replay to work on your machine. When working on the \textit{aonl} machines, can omit the all but \textit{DB\_DIR}.

\begin{enumerate}
\item ANALYZER - Set to the directory where the analyzer lives. Can omit on \textit{aonl} machines.
\item PATH - Add the directory where the analyzer lives. Can omit on \textit{aonl} machines.
\item LD\_LIBRARY\_PATH - Add the directory where the analyzer lives. Can omit on \textit{aonl} machines.
\item DB\_DIR - Set to the directory of the replay database.
\end{enumerate}

If you are using the official replay, a script is provided to do this for you. In the replay directory, run:
\\\\
\texttt{source sourceme.sh}
\\\\
This script can serve as an example for setting up your own installation as well. If you are using \texttt{csh} instead of \texttt{bash} be sure to replace \texttt{export} with \texttt{setenv} and \texttt{=} with a space.

\section{Directory Structure}
The code has been set up so that the number of lines that need to be changed to reflect your directory structure can be kept to a minimum.
\\\\
\noindent In rootlogon.C:
\begin{enumerate}
\item Set \textit{char* replay\_dir\_prefix} to the directory where your code lives. There must be a trailing \textit{/\%s}. In most setups, leaving it as \textit{./\%s} should suffice.
\end{enumerate}
In def\_tritium.h:
\begin{enumerate}
\setcounter{enumi}{2}
\item Set \textit{char* REPLAY\_DIR\_PREFIX} to the directory where your code lives. There must be a trailing \textit{/\%s}. In most setups, leaving it as \textit{./\%s} should suffice.
\item Set \textit{char* ROOTFILE\_DIR\_PREFIX} to the directory where your root files will be stored. There must be a trailing \textit{/\%s}.
\item Add the location of your raw data folder to the \textit{static const char* PATHS[]} array if you do not have access to the standard directories.
\end{enumerate}

Now you must compile the ReplayCore. To do this, type the following commands into a terminal in the replay directory.
\\\\
\texttt{analyzer\\
.L ReplayCore64.C+\\
.q}

\section{Compiling The Libraries}

To compile all of the libraries, a script has been provided. Go into the \textit{replay/libraries/} directory. From there, run the command:\\\\
\texttt{./libs.sh}

\chapter{Running The Replay Code}

To run the replay code, navigate to the replay directory. Once there, type:
\\\\
\texttt{analyzer replay\_tritium.C}
\\

The analyzer will then prompt you for a run number, followed by the number of events to replay. The output root file will be stored wherever you defined \textit{char* ROOTFILE\_DIR\_PREFIX} in def\_tritium.h. In the official replay installation, this location is \textit{/chafs1/work1/tritium/Rootfiles/}.
\\

To load the contents of the root file into an analyzer session:
\\\\
\texttt{analyzer [path\_to\_root\_file]}
\\

To view the online plots:
\\\\
\texttt{onlineTritium [run\_number]}
\\\\

\end{document}
